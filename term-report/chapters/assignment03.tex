%%%----------------------------------------------------------
\chapter{Iterative Closest-Point Algorithm}
%%%----------------------------------------------------------

This assignment took a step further and instead of trying to determine a point correspondence in a binary dot image the task was to find a reasonable correspondence between a pair of stereo pictures.
The \textit{Harris Corner Detector}\cite{Harris2020} was used to detect corners in the images. Then a point correspondence between the corners in the left and the right image needed to be determined. Both pictures were taken from a slightly different viewpoint and the ambient conditions may differ so the amount of detected corners may differ. This results in the possibility that not all corners of one picture have a matching corner in the other picture, which causes the problem of outliers. 


\section{Algorithm}

Prepare images- part into two point clouds. (corner strength settings)

\begin{enumerate}
	\item Use Corner Detector
	\item Initial Transformation
	\item ICP - Affine Fitter -> Transformation Matrix
	\item Find point pairs
	\item Transform back then connect point pairs with lines
\end{enumerate}

\begin{algorithm}
User Corner Detector

Separate into two point clouds

Get centroids of both point clouds

Start ICP with vector connecting centroids as initial transformation



\end{algorithm}
