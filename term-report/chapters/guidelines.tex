%%%--- Remove the preface from your report in not needed ----
\chapter*{Guidelines for authoring lab reports}
\addcontentsline{toc}{chapter}{Guidelines}
\chaptermark{Guidelines}
%%%----------------------------------------------------------

\section*{Cumulative lab report}

The lab report is a \textbf{single, cumulative document} which should contain 
a concise and well-structured summary of the work you did in this course.
Also, you are asked to demonstrate and discuss your ``report in progress'' at any time
throughout the semester.
If help or advice is needed, please ask in class or use the course's online forum.


\section*{Weekly and final submissions}

You are asked to upload a snapshot of your worked-out assignments weekly (\ie, prior to the next lab unit). 
These submissions are not graded but randomly checked to verify your progress.
The final (complete) documentation for all assignments must be turned in at 
the end of the semester, prior to the (oral) exam. 
Thus you can pace your work individually and turn back to previous assignments 
for improvements at any later point.
%
\begin{quote}
\textbf{Note} that this \textbf{freedom} puts a lot of \textbf{responsibility} on yourself. Make sure
that you start to write immediately, make steady progress and nothing important is left behind!
\end{quote}


\section*{Document structure and content}

This document should help you to get started with the report.
It is strongly suggested to use the final format right away to avoid
surprises at a later point. Also, you will discover that writing and documenting your findings  
can help you in developing good and understandable solutions from the very beginning.
Here are a few hints for writing your reports:
%
\begin{itemize}
\item
One \textbf{chapter} should be dedicated to each \textbf{assignment} (note that chapter names have been modified for this).
\item
Make notes and write down your concepts immediately, that is, \textbf{before} you start coding!
\item
Describe each given task in your own words (do not just replicate the assignment). 
Then describe your approach, explain the main difficulties, clearly outline your solution, finally 
provide illustrative and meaningful results. 
\item
Try to go beyond the material you find elsewhere, use and extend formal (mathematical) descriptions in a creative way. 
Also, try to keep your notation simple and consistent, which is not always easy to do. Look at good examples 
and consider this part of the learning process.
\item
Be careful and creative when it comes to designing meaningful tests and selecting examples.
Do not make screenshots but save the relevant images with ImageJ (usually as PNGs).
\item
Always give appropriate references to literature, figures and other work you used!
\item
Get used to work with formal and concise descriptions (math, symbols, relations, algorithms, \ldots) and train yourself in 
``getting the notation right''. 
\item 
Write in complete sentences and try to use a ``professional'' language.
\end{itemize}


\section*{The bad and the ugly}

\begin{itemize}
\item
\textbf{Don't just show program code!}
Use prose with mathematical and algorithmic notation wherever appropriate (use the assignments
and lecture notes for guidance).
Insert actual code sparingly and only to show particularly interesting or critical parts 
of your implementation.
\item
\textbf{Do not explain details that are trivial} or elementary (such as Pythagoras' law, 
for example). Otherwise, make a reference to the \textit{all} sources 
you used (including school books, blogs, WikiPedia \etc).
\item
\textbf{Do not just replicate} equations and figures from the lecture materials, but
-- as said above -- describe the task in your own words. 
In particular, you will be \textbf{executed} (\ie, beheaded, drawn and quartered) if you
ever copy/paste equations from the assignment or
any other sources. Make sure you write these things yourself (that's what LaTeX is 
famous for)!
\end{itemize}